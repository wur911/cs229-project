% CS 229 Project Paper
% This is for the milestone; though we're expanding it for the final draft,
% it will probably be useful to have around afterwards.

% Goal: fill three pages with useful background data, plans, stuff.

\documentclass{amsart}
% At this point, my macros file is a bit disorganized. But it ought to be
% usable.

% For page margins
\usepackage[margin=1in]{geometry}
\usepackage{xcolor}
\usepackage{graphicx}
\usepackage{mathtools}
\usepackage{enumerate}
% Sets font to Garamond. Feel free to change this or remove it.
% \usepackage[garamond]{mathdesign}
\usepackage{hyperref}
% Makes figure labels work better
\usepackage[all]{hypcap}

% Note: we might want to use fancyhdr. We can worry about that later.
\pagestyle{plain}

\newcommand{\N}{\mathbb N}
\newcommand{\Z}{\mathbb Z}
\newcommand{\Q}{\mathbb Q}
\newcommand{\R}{\mathbb R}
\newcommand{\T}{^{\mathrm T}\!}
\newcommand{\ud}{\,\mathrm d}
\newcommand{\dfr}[2]{\frac{\mathrm d #1}{\mathrm d #2}}
\newcommand{\pfr}[2]{\frac{\partial #1}{\partial #2}}
\DeclareMathOperator*{\argmax}{arg\,max}
\DeclareMathOperator*{\argmin}{arg\,min}

% I like bold vectors, and this allows bold Greek letters too.
% Feel free to change or remove this.
\renewcommand{\vec}[1]{\boldsymbol{\mathbf{#1}}}

% So that we have \paren{...} instead of \left( and \right)
\DeclarePairedDelimiter\paren{(}{)}
\DeclarePairedDelimiter\ang{\langle}{\rangle}
\DeclarePairedDelimiter\abs{\lvert}{\rvert}
\DeclarePairedDelimiter\norm{\lVert}{\rVert}
% Swap paren* and paren, etc., so that the normal version resizes by default.
% Meanwhile, one can use \paren*[\Big]{...} to customize the size easily.
% It would be interesting to wrap this up into a custom \definedelimiter command...
\makeatletter
    \let\oldparen\paren
    \def\paren{\@ifstar{\oldparen}{\oldparen*}}
    \let\oldang\ang
    \def\ang{\@ifstar{\oldang}{\oldang*}}
    \let\oldabs\abs
    \def\abs{\@ifstar{\oldabs}{\oldabs*}}
    \let\oldnorm\norm
    \def\norm{\@ifstar{\oldnorm}{\oldnorm*}}
\makeatother

% This allows x"i -> x^{(i)} and x"{i+1} -> x^{(i+1)}
\catcode`\"=13
\newcommand{"}[1]{^{(#1)}}

\newcommand{\e}{\varepsilon}
\newcommand{\E}[1]{\cdot 10^{#1}}

\begin{document}
\title{Astronomical Implications of Machine Learning}
\author{
	\lowercase{\href{mailto:adebray@stanford.edu?subject=CS\%20229\%20Project}}{Arun Debray}\\
	\lowercase{\href{mailto:wur911@stanford.edu?subject=CS\%20229\%20Project}}{Raymond Wu}\\
	\today
}
\maketitle

% Do we need a table of contents? Probably not.
%\tableofcontents

% I started writing things. You are welcome to edit them!

% Abstract might not be necessary
\begin{abstract}
In this project we aim to use supervised learning to develop a classifier for stellar lightcurves
 to detect whether they demonstrate the existence of exosolar planets.
\end{abstract}
\section{Introduction}
% What are exoplanets? How are they detected?
The recent discoveries of planets around stars other than our own is among the most significant trends in astronomy today. Long debated by philosophers and physicists alike, no such planets were known until 1992, when two planets were discovered around a star called PSR B1257+12. In the two decades since then, over a thousand such planets have been discovered, diverse in many ways. Thanks to these discoveries, astronomers are learning more about planetary systems other than our own, responding to these questions about other solar systems and even how probable Earth-like life could be in the universe.

In this paper, we will use the following standard terminology.
\begin{itemize}
	\item An \emph{exosolar planet} is defined to be a planet that orbits a star other than the Sun.\footnote{The words \emph{exoplanet}, \emph{exosolar planet}, and \emph{extrasolar planet} all mean the same thing, and are used interchangeably.} The standard definition of a planet has two kinds of ambiguity: very low-mass objects in our solar system, such as Pluto, were defined to be ``dwarf planets,'' and the boundaries of this definition aren't entirely clear. Very high-mass planets, however, resemble very small stars; though they don't undergo hydrogen fusion, they look very much like a dim type of star called a brown dwarf. The boundary is somewhat arbitrarily delineated at 14 Jupiter masses. However, neither of these is a great concern in this paper: science is yet unable to detect Pluto-sized worlds around another star, so the low-mass ambiguity does not arise in this data, and the high-mass boundary is not as important: a classifier that discovers planets and small brown dwarfs is still useful. However, it will be helpful to distinguish these systems from eclipsing binaries (see below).
	\item \emph{Planetary transit} is a method of exoplanet detection, In general, because planets are very dim relative to their bright host stars, they cannot be directly imaged, in the same way that it is difficult to detect a firefly near a searchlight from afar. Thus, several indirect methods exist. Planetary transit repeatedly checks the brightness of a star over time; periodic, regular dips in this output sometimes happen because an exoplanet crosses between its sun and the observer. Thus, a planet may be detected without direct observation.
	\item A \emph{lightcurve} is a graph of a star's brightness over time. A transiting exoplanet will thus manifest itself as a lightcurve that is relatively constant, but with regular, small dips corresponding to the transits.%TODO add graph
	 The brightness is often given in units of magnitude rather than strict luminosity, because this runs over a relatively nicer range of values.
	\item An \emph{eclipsing binary} is a pair of stars that orbit each other, but such that each eclipses the other from the Earth's point of view during the orbit. These generally don't contain transiting exoplanets, but instead form an important negative example. Their lightcurves look like those of exoplanets, but they aren't exoplanets.
\end{itemize}
% Why Kepler?

Until recently, most exoplanets weren't detected by transit; astronomers used any of several other methods to find them. However, when the Kepler telescope was launched, it provided a wealth of data about transiting exoplanets, in particular showing that many stars could be surveyed at once. Since Kepler provided such a wealth of data about exoplanets, we decided to try to train a classifier on its lightcurves.%expand? TODO
% Planet Hunters, maybe

\section{Methodology}

% and so on.
\section{Results}

\section{Analysis}

\section{Future Work}

% bibliography of some sort
\end{document}
